\section{Liczby Stirlinga I i II rodzaju}

\begin{definition}[liczby Stirlinga II rodzaju]
	Symbol
	\begin{equation}
		\stirlingtwo{n}{k}
	\end{equation}
	oznacza liczbę sposobów podziału zbioru $n$ elementowego na $k$ niepustych podzbiorów.
\end{definition}

Mamy $\stirlingtwo{n}{n} = \stirlingtwo{n}{1} = 1$, nawet kiedy $n = 0$.
Z drugiej strony $\stirlingtwo{n}{0} = [n = 0]$, jak pisze Knuth ,,przypadek (...) wymaga nieco zręczności''.
Inne szczególne wartości to $\stirlingtwo{n}{2} = 2^{n-1}-1$ (pierwszy zbiór składa się z $n$ i dowolnych ale nie wszystkich liczb spośród $\{1, 2, ..., n-1\}$) oraz $\stirlingtwo{n}{n-1} = {n \choose 2}$ (jeden zbiór składa się z dwóch elementów, pozostałe są singletonami).


\begin{definition}[liczby Stirlinga I rodzaju]
	Symbol
	\begin{equation}
		\stirlingone{n}{k}
	\end{equation}
	oznacza liczbę sposobów podziału zbioru $n$ elementowego na $k$ niepustych cykli (,,naszyjników'').
\end{definition}

Istnieje jeden 1-cykl $[1]$, jeden 2-cykl $[1, 2] = [2, 1]$, ale dwa 3-cykle $[1, 2, 3] \neq [1, 3, 2]$.
Łatwo widać, że w ogólności mamy $(n-1)!$ różnych $n$-cykli; zatem $\stirlingone{n}{1} = (n-1)!$ dla $n > 0$, $\stirlingone{n}{n} = 1$ oraz $\stirlingone{n}{n-1} = {n \choose 2}$.

Ponieważ każdą permutację można rozłożyć jednoznacznie na cykle, mamy też
\begin{equation}
	\sum_{k=0}^n \stirlingone{n}{k} = n!.
\end{equation}


Jak nietrudno się domyślić, podobieństwa w definicjach liczb Stirlinga I i II rodzaju sprawiają, że ich własności także są podobne.
Prawdziwe są na przykład proste rekurencje
\begin{align}
	\stirlingtwo{n}{k} & = \stirlingtwo{n-1}{k-1} + k \stirlingtwo{n-1}{k}, \\
	\stirlingone{n}{k} & = \stirlingone{n-1}{k-1} + (n-1) \stirlingone{n-1}{k}
\end{align}
ponieważ element $n$ albo mieścimy w jednym z $k$ podzbiorów zbioru $\{1, 2, \ldots, n-1\}$, albo będzie jedynym elementem nowego zbioru.

Zachodzi

\begin{proposition}[przekształcanie potęg]
\begin{align}
x^n & = \sum_{k} \stirlingtwo{n}{k} x\fallingfactorial{k} 
      = \sum_{k} \stirlingtwo{n}{k} (-1)^{n-k} x\raisingfactorial{k}, \\
x\fallingfactorial{n} & = \sum_{k} \stirlingone{n}{k} (-1)^{n-k} x^k, \\
x\raisingfactorial{n} & = \sum_{k} \stirlingone{n}{k} x^k.
\end{align}
\end{proposition}

o czym łatwo przekonać się indukcyjnie, patrz \cite[s. 294]{knuth}. % TODO: 

\begin{proposition}[wzory inwersji]
\begin{align}
	\sum_k \stirlingtwo{n}{k}\stirlingone{k}{m}(-1)^{n-k} & = [m = n] \\
	\sum_k \stirlingone{n}{k}\stirlingtwo{k}{m}(-1)^{n-k} & = [m = n].
\end{align}
\end{proposition}

Liczby Stirlinga II rodzaju są związane z rozkładem Poissona:

\begin{proposition}[wzór Dobińskiego]
\begin{equation}
	B_n(x) = \sum_{k= 0}^n \stirlingtwo{n}{k} x^k
\end{equation}
\end{proposition}

DO-ZROBIENIA
$$S(n,k)=1/(k!)\sum_(i=0)^k(-1)^i(k; i)(k-i)^n$$, 

$$x^n	=	sum_(m=0)^(n)S(n,m)(x)_m	
(11)
	=	sum_(m=0)^(n)S(n,m)x(x-1)...(x-m+1),$$

	 $$sum_(n=k)^inftyS(n,k)(x^n)/(n!)=1/(k!)(e^x-1)^k, 	
(13)
and$$


$$
sum_(n=1)^(infty)S(n,k)x^n	=	sum_(n=k)^(infty)S(n,k)x^n	
(14)
	=	(x^k)/((1-x)(1-2x)...(1-kx))	
(15)
	=	((-1)^k)/(((x-1)/x)_k)
$$

Stirling numbers of the second kind are intimately connected with the Poisson distribution through Dobiński's formula
$$
 B_n(x)=sum_(k=0)^nx^kS(n,k) 	
(18)$$
where Bn(x) is a Bell polynomial.