
\documentclass{createspace}


\newcommand{\N}{\mathbb N}
\newcommand{\Z}{\mathbb Z}
\newcommand{\Q}{\mathbb Q}
\newcommand{\R}{\mathbb R}
\newcommand{\C}{\mathbb C}

% https://tex.stackexchange.com/questions/86056/how-to-write-stirling-numbers-of-the-second-kind
\DeclareRobustCommand{\stirlingtwo}{\genfrac\{\}{0pt}{}}
\DeclareRobustCommand{\stirlingone}{\genfrac[]{0pt}{}}

\newcommand{\fallingfactorial}[1]{^{\underline{#1}}}
\newcommand{\raisingfactorial}[1]{^{\overline{#1}}}

\usepackage{comment}
\includecomment{comment}

% forced by -output-directory option in makefile
% see https://tex.stackexchange.com/a/564296
\usepackage{etoolbox}
\makeatletter
\patchcmd\imki@putindex
  {\imki@exec{\imki@program \imki@options #1.idx}}
  {\imki@exec{cd ../build;\imki@program\imki@options#1.idx}}
  {\message{Patch succeeded in imki@putindex}}
  {\errmessage{Patch failed in imki@putindex}}
\makeatother
\makeindex[title=Skorowidz]
\makeindex[name=persons,title=Indeks osób]

\usepackage{enumitem}
\usepackage{booktabs}
\usepackage{longtable}
\usepackage[table]{xcolor}
\usepackage[colorinlistoftodos,prependcaption]{todonotes}
\usepackage{tikz}
\usetikzlibrary{arrows.meta}
\usetikzlibrary{decorations.markings}
\usetikzlibrary{decorations.pathreplacing}
\usetikzlibrary{knots}
\colorlet{darkblue}{blue!80!black}
\definecolor{diagramfiller}{HTML}{5784BA}
\definecolor{first_colour}{HTML}{A15D98}

\definecolor{lightgray}{gray}{0.9} % define lightgray
\let\oldtabular\tabular % alternate rowcolors for all tables
\let\endoldtabular\endtabular
\renewenvironment{tabular}
{\rowcolors{2}{white}{lightgray}\oldtabular}
{\endoldtabular}
\let\oldlongtable\longtable % alternate rowcolors for all long-tables
\let\endoldlongtable\endlongtable
\renewenvironment{longtable}
{\rowcolors{2}{white}{lightgray}\oldlongtable}
{\endoldlongtable}

\author{Ana Zlatić}
\title{Kombinatoryka}

\begin{document}



% strona pierwsza

\thispagestyle{empty}
{\noindent\fontsize{18pt}{18pt}\selectfont Księgozbiór matemagiczny, tom 22}

\noindent\makebox[\linewidth]{\rule{\textwidth}{1pt}}

\newpage

% koniec strony pierwszej



% strona druga

\thispagestyle{empty}
\phantom{nothing}
\newpage

% koniec strony drugiej



% strona trzecia

\thispagestyle{empty}
{\noindent\fontsize{18pt}{18pt}\selectfont Ana Zlatić}

\noindent\makebox[\linewidth]{\rule{\textwidth}{1pt}}

\vspace{10mm}

{\noindent\fontsize{24pt}{24pt}\selectfont \textbf{Kombinatoryka}}
\vspace{10mm}

{\noindent\fontsize{14pt}{14pt}\selectfont Wydanie pierwsze}

\newpage

% koniec strony trzeciej



% strona czwarta

\thispagestyle{empty}
\begin{figure}[H]
\begin{minipage}[b]{.48\linewidth}
{\noindent Ana Zlatić\\
Studentski trg 16,\\
11158 Beograd, Serbia}
\end{minipage}
%\begin{minipage}[b]{.48\linewidth}
% {\noindent imię nazwisko\\
% nazwa szkoły\\
% ulica\\
% kod, miasto, kraj}
% \end{minipage}
\end{figure}

{\noindent \textbf{Kategorie MSC 2020}\\05A (enumerative combinatorics)} \vspace{5mm}

{\noindent \textbf{Tytuł oryginału}\\Kombinatorna matematika}
\vspace{5mm}

{\noindent \textbf{Z serbsko-chorwackiego tłumaczyła}\\... ...} 
\vspace{5mm}

{\noindent \textbf{Okładkę zaprojektował}\\... ...}
\vspace{5mm}

{\noindent \textbf{Zredagował}\\... ...}
\vspace{5mm}

{\noindent \textbf{Zredagowała technicznie}\\... ...}
\vspace{5mm}

{\noindent \textbf{Złożyli i połamali}\\... ...}
\vspace{5mm}

{\noindent \textbf{Korekty dokonali}\\... ...}

\vfill

{\noindent Copyleft by Antykwariat Czarnoksięski, Gorzów Wielkopolski 2023.
Książka, a także każda jej część, mogą być przedrukowywane oraz w jakikolwiek inny sposób reprodukowane czy powielane mechanicznie, fotooptycznie, zapisywane elektronicznie lub magnetycznie, oraz odczytywane w środkach publicznego przekazu bez pisemnej zgody wydawcy.}

\vspace{5mm}

{\noindent Przygotowano w systemie \TeX, wydrukowano na siarczystym papierze.}

% koniec strony czwartej


\tableofcontents

\chapter{Preludium}
Liczby Stirlinga wprowadził James Stirling w \cite{stirling1753}.
\index[persons]{Stirling, James}
\index{liczby Stirlinga|(}

\section{Liczby Stirlinga II rodzaju}

\begin{definition}[liczby Stirlinga II rodzaju]
	Symbol
	\begin{equation}
		\stirlingtwo{n}{k}
	\end{equation}
	oznacza liczbę sposobów podziału zbioru $n$ elementowego na $k$ niepustych podzbiorów.
\end{definition}

Mamy $\stirlingtwo{n}{n} = \stirlingtwo{n}{1} = 1$, nawet kiedy $n = 0$.
Z drugiej strony $\stirlingtwo{n}{0} = [n = 0]$, jak pisze Knuth ,,przypadek (...) wymaga nieco zręczności''.
Inne szczególne wartości to $\stirlingtwo{n}{2} = 2^{n-1}-1$ (pierwszy zbiór składa się z $n$ i dowolnych ale nie wszystkich liczb spośród $\{1, 2, ..., n-1\}$) oraz $\stirlingtwo{n}{n-1} = {n \choose 2}$ (jeden zbiór składa się z dwóch elementów, pozostałe są singletonami).

Liczby Stirlinga II rodzaju spełniają prostą rekurencję
\begin{equation}
	\stirlingtwo{n}{k} = k \stirlingtwo{n-1}{k} + \stirlingtwo{n-1}{k-1},
\end{equation}
ponieważ element $n$ albo mieścimy w jednym z $k$ podzbiorów zbioru $\{1, 2, \ldots, n-1\}$, albo będzie jedynym elementem nowego zbioru.

Mamy też
\begin{equation}
x^n = \sum_{k} \stirlingtwo{n}{k} x\fallingfactorial{k},
\end{equation}

o czym łatwo przekonać się indukcyjnie, patrz \cite[s. 294]{knuth}. % TODO: Knuth 


DO-ZROBIENIA
$$S(n,k)=1/(k!)\sum_(i=0)^k(-1)^i(k; i)(k-i)^n$$, 

$$x^n	=	sum_(m=0)^(n)S(n,m)(x)_m	
(11)
	=	sum_(m=0)^(n)S(n,m)x(x-1)...(x-m+1),$$

	 $$sum_(n=k)^inftyS(n,k)(x^n)/(n!)=1/(k!)(e^x-1)^k, 	
(13)
and$$


$$
sum_(n=1)^(infty)S(n,k)x^n	=	sum_(n=k)^(infty)S(n,k)x^n	
(14)
	=	(x^k)/((1-x)(1-2x)...(1-kx))	
(15)
	=	((-1)^k)/(((x-1)/x)_k)
$$

Stirling numbers of the second kind are intimately connected with the Poisson distribution through Dobiński's formula
$$
 B_n(x)=sum_(k=0)^nx^kS(n,k) 	
(18)$$
where Bn(x) is a Bell polynomial.



\newpage
test
\newpage
test
\index{liczby Stirlinga|)}



\raggedright
\bibliographystyle{plain}
\bibliography{combinatoire}

\indexprologue{\small Prolog zwykły...}
\printindex

\indexprologue{\small Prolog nazwisk...}
\printindex[persons]

\end{document}

